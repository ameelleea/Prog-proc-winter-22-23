\documentclass{article}
\usepackage[utf8]{inputenc}

\title{Programma per la verifica delle congetture di Beal, Collatz e Cramer}
\date{Sessione invernale 2022/23}
\author{Milena Balducci \\ Matricola 321791}

\begin{document}

\maketitle

\newpage

\section{Specifica del problema}
La congettura di Beal asserisce che se $a^x + b^y = c^z$ dove $a,b,c,x,y,z, \in \mathbb{N}$ con $a,b,c\ge1$ e $x,y,z\ge3$, allora $a,b,c$ hanno un fattore primo in comune. 
La congettura di Collatz asserisce che la funzione $f:\mathbb{N}_{>0}\longrightarrow\mathbb{N}_{>0}$ definita ponendo $f(n) = n/2$ se $n$ è pari e $f(n) = 3 * n + 1$ 
se $n$ è dispari genera 1 dopo un numero finito di applicazioni ai numeri mano a mano ottenuti. 
La congettura di Cramer asserisce che il valore assoluto della differenza tra due numeri primi consecutivi è minore del quadrato del logaritmo naturale del più piccolo dei due numeri. 
Scrivere un programma ANSI C che chieda all'utente quale congettura intende considerare e poi la verifica acquisendo dalla tastiera $a,b,c,x,y,z$ nel primo caso 
(se non vale  $a^x + b^y = c^z$, il programma lo stampa sullo schermo e poi verifica comunque se $a,b,c$ hanno un fattore primo in comune e ne stampa l'esito 
sullo schermo), $n>0$ nel secondo caso (il programma stampa sullo schermo tutti i numeri generati), 
due numeri primi consecutivi nel terzo caso (il programma stampa sullo schermo sia il valore assoluto della differenza tra i due numeri che il quadrato del 
logaritmo naturale del più piccolo dei due numeri). 

\newpage

\section{Analisi del Problema}
\subsection{Dati di Ingresso del Problema}
I dati di ingresso sono l'operazione scelta dall'utente e, a seconda della scelta effettuata:

\begin{itemize}
\item tre potenze con base intera $b\ge1$ e esponente intero $e\ge3$;
\item un numero intero n$>0$;
\item due numeri primi $n_1$ e $n_2$ consecutivi.
\end{itemize}


\subsection{Dati di uscita del Problema}
I dati di uscita del problema sono l'esito della verifica della congettura in esame e, in base all'operazione scelta dall'utente:

\begin{itemize}
\item l'insieme dei fattori primi comuni alle tre basi scelte; 
\item l'insieme dei valori generati dalle applicazioni successive della funzione $f:\mathbb{N}_{>0}\longrightarrow\mathbb{N}_{>0}$; 
\item il valore assoluto della differenza tra i due numeri primi inseriti, e il quadrato del logaritmo naturale del minore dei due. 
\end{itemize}

\subsection{Relazioni intercorrenti tra i dati del Problema}
I fattori primi di un numero $x \in \mathbb{N}$ sono tutti i numeri primi $n \in \mathcal{P}:={n \in \mathbb{N}:n=n*1}$ che lo dividono esattamente (ovvero senza resto).
L'insieme $F(x)$ dei fattori primi di un numero $x$ sono definibili a partire da $n=2$ nel seguente modo:
\begin{itemize}
    \item Se x/n \in \mathbb{N}, $n \in F(x)$;
    \item Se x/n \notin \mathbb{N}, $n = n+1$.
\end{itemize}  

La congettura di Collatz consiste nell'applicare la funzione $f:\mathbb{N}_{>0}\longrightarrow\mathbb{N}_{>0}$ al risultato della precedente applicazione, ottenendo così
un insieme di numeri ${n_0, n_1,....n_{k-1}, n_k}$, con $n_k = 1$ come indicato nella specifica del problema. 
E' possibile implementare ricorsivamente questa funzione distinguendo due casi:
\being{itemize}
\item Caso base: $k = 0 \Rightarrow n_0 = n$
\item Caso generale: $k > 0 \Rightarrow n_k = f(n_{k-1})$, 
\end{itemize}
dove $f(n)=\begin{cases} n/2, & \mbox{se }n\mbox{ pari} \\ 3n+1, & \mbox{se }n\mbox{ dispari}\end{cases}$

\newpage


\section{Progettazione dell'Algoritmo}

\subsection{Scelte di progetto}
Il numero di dati di input del problema varia a seconda dell'operazione scelta dall'utente. Per questo motivo, onde evitare di dichiarare variabili in eccesso che
non verrebbero utilizzate, è utile impostare i valori di input come una array di valori numerici, la cui dimensione viene allocata dinamicamente in base all'operazione
scelta dall'utente.

La congettura di Collatz si presta naturalmente ad essere implementata tramite una funzione ricorsiva, che ha come condizione di terminazione $n_k = 1$, con $k\ne0$.
I numeri ottenuti dall'esecuzione della funzione vengono allocati in una array dinamica, la cui dimensione viene calcolata contestualmente all'esecuzione della funzione.

Per evitare ridondanze di codice, si decide di sviluppare dei sottoprogrammi per effettuare operazioni che vengono ripetute numerose volte all'interno dell'algoritmo, come
la verifica dell'appartenenza di un numero all'insieme dei numeri primi, la scomposizione in fattori primi o l'acquisizione e validazione stretta degli input.


\subsection{Passi dell'algoritmo}
I passi dell'algoritmo per risolvere il problema sono i seguenti:

\begin{itemize}
\item Acquisire l'operazione scelta dall'utente.
\item In base alla scelta effettuata:
    \begin{itemize}
    \item Allocare la dimensione dell'array, che conterrà sei elementi (tre basi, tre esponenti);
    \item Acquisire le tre potenze $a^x, b^y, c^z$;
    \item Verificare se la relazione $a^x + b^y = c^z$ è valida per $a,b,c,x,y,z$:
        \begin{itemize}
        \item Se la relazione vale per i tre valori acquisiti, cercare i fattori primi comuni;
        \item Se la relazione non vale per i tre valori acquisiti, verificare se sono comunque presenti fattori primi comuni. 
        \end{itemize}
    \item Stampare a schermo gli eventuali fattori primi individuati.
    \end{itemize}

    \item oppure:

    \begin{itemize}
    \item Allocare la dimensione dell'array, che conterrà un solo elemento;
    \item Acquisire il valore intero maggiore di zero;
    \item Assegnare il valore come parametro alla funzione ricorsiva;
    \item Stampare la sequenza di valori ottenuti;
    
    \end{itemize}

    \item oppure:
    \begin{itemize}
        \item Allocare la dimensione dell'array, che conterrà due elementi;
        \item Acquisire e validare i due numeri primi consecutivi;
        \item Calcolare la differenza dei due numeri, e trasformarla in valore assoluto;
        \item Verificare quale dei due numeri acquisiti è il minore;
        \item Calcolare la potenza quadrata del logaritmo naturale del numero minore;
        \item Verificare se il valore assoluto della differenza è minore di $(ln(min(p_n, p_{n+1})))^2$;
        \item Stampare i due valori ottenuti;
    \end{itemize}

    \item Stampare l'esito della verifica.
\end{itemize}


\end{document}
